\documentclass[12pt]{article}
\usepackage[english]{babel}
\usepackage[utf8x]{inputenc}
\usepackage{amsmath}
\usepackage{graphicx}
\usepackage{algorithm}
\usepackage[noend]{algpseudocode}
\usepackage{changepage}
\usepackage{color}
\usepackage{siunitx}
\usepackage{booktabs}

\usepackage[colorinlistoftodos]{todonotes}
\usepackage{hyperref}

\begin{document}

	\begin{titlepage}
		
		\newcommand{\HRule}{\rule{\linewidth}{0.5mm}} % 
		\center % Center everything on the page
		 
		\textsc{\LARGE Indian Institute of Technology, Kanpur}\\[1.5cm] 
		
		
		\HRule \\[0.4cm]
		{ \huge \bfseries Important Concepts}\\[0.4cm] % Title of your document
		\HRule \\[1.5cm]
		 
		\Large \emph{Author:}\\
		Shubham Agrawal\\[3cm] % Your name
		
		
		{\large \today}\\[2cm] % Date, change the \today to a set date if you want to be precise
		
		\vfill % Fill the rest of the page with whitespace
	
	\end{titlepage}
	
	
	\begin{abstract}
	There are various concepts in probability theory and linear algebra that we keep on forgetting and they keep haunting us throughout our lifetime. I attempt to list and explain a few of them, which I myself need very much. 
	\end{abstract}
	
	\section {Vector Spaces}
		Vector Spaces are defined with respect to (set of vectors $V$, scalar multiplication $.$ over $F$ and vector addition $+$). A set of simple mathemetical objects form a vector space when their operations $+$ and $.$ follow following axioms:
		\begin{gather*}
			+: V \times V \rightarrow V \\
			.: F \times V \rightarrow V 
		\end{gather*}
		% Please add the following required packages to your document preamble:
		% \usepackage{graphicx}
		\begin{table}[H]
			\centering
			\caption{Vector Space Properties}
			\label{my-label}
			\resizebox{\textwidth}{!}{%
				\begin{tabular}{|l|l|}
					\hline
					\multicolumn{1}{|c|}{\textbf{$+$}} & \multicolumn{1}{c|}{\textbf{$.$}} \\ \hline
					Associativity $u + (v+w) = (u+v)+w$ & Compatibility $a(bv) = (ab)v$ \\ \hline
					Commutativity & \begin{tabular}[c]{@{}l@{}}Identity element fo scalar multiplication\\ I.v = v\end{tabular} \\ \hline
					Identity Element  $\exists 0 \in V$ $v+0 = v \forall v \in V$ & $a(u+v) = au + av$ \\ \hline
					Inverse Element for every $v \in V, \exists v^{-1} s.t v + v^{-1} = 0$ & $(a+b)v = av + bv$ \\ \hline
				\end{tabular}%
			}
		\end{table}
		
	\section {Subspaces}
	Let $W$ be a subset of vector space $V$ (over field $K$). Then $W$ is called subspace if:
	\begin{itemize}
		\item {$0 \in W$}
		\item {$\forall u,v \in W \implies u+v \in W$}
		\item {$\forall c \in K$ and $w \in W \implies c.w \in W$}
	\end{itemize}
	\section {Linear Transformation}: 
		Mapping between two vector spaces $V$ and $W$ (usually over same field $F$) such that operation is preserved. That is, applying the mapping before or after the operation should not matter. 
		\begin{gather*}
			f(u+v) = f(u) + f(v) \\
			f(cu) = cf(u)
		\end{gather*}

	\section{Eigen vectors and Eigen Values}
		Eigen Vector and Eigen values are defined with respect to Linear transformation and corresponding vector spaces. A vector is called an eigen vector wrt a particular linear trasformation if it is only scaled by a fixed value (called eigen value) on applying the linear transformation.
		\begin{gather*}
			T(\textbf{v}) = \lambda \textbf{v}
		\end{gather*}
		There is a correspondence between n by n square matrices and linear transformations from an n-dimensional vector space to itself. For this reason, it is equivalent to define eigenvalues and eigenvectors using either the language of matrices or the language of linear transformations.
		
		For a matrix, eigenvalues and eigenvectors can be used to decompose the matrix, for example by diagonalizing it.
	\section {Matrix Similarity}
	In linear algebra, two $n$-by-$n$ matrices $A$ and $B$ are called similar if 
	\begin{gather*}
		B = P^{-1}AP
	\end{gather*}
	for some invertible $n$-by-$n$ matrix $P$.
	\section {Diagonalizable Matrix}
	A matrix is called diagonalizable, if it's similar to a diagonal matrix.
	
	These matrices are of interest because of their ease of handling.
	\section{Eigen Value decomposition}
	
	\section{Singular value decomposition}
	\section{Gram Schmidt Orthogonalization}
	\section{Positive Definite Matrix}
	
		A symmetric (Hermitian) $n \times n$ real ( complex) matrix $M$ is said to be Positive Definite if $z^TMz > 0$ for every non-zero vector $z$. In case of positive semidefinite, $z^TMz \ge 0$. 
		
		\subsection{Properties of Positive definite matrix}
			Let $M$ be an $n \times n$ symmetric matrix.
			\begin{itemize}
				\item It has positive eigen values. Vice-versa, if a matrix has all it's eigenvalues real and positive, the matrix is positive definite. 
				\item For any real invertible matrix $A$, \textbf{$A^{T}A$ is always positive definite}
				\item Positive definite matrix has \textbf{positive determinant}. This means, that PD matrix is always \textbf{nonsingular}
			\end{itemize}
	\section{Covariance Matrix}
	
		Let $\textbf{X} = \{X_1, X_2 \cdots X_n\}$ be a random vector. Then, 
		\begin{equation}
		Cov(\textbf{X}) = [\Sigma]_{n \times n} 
		\end{equation}
		where $\Sigma_{ij} = cov(X_i, X_j) = E((X_i - \mu_i)(X_j - \mu_j))$
		\\
		Also, Let $\textbf{Y} = \{Y_1, Y_2 \cdots Y_m\}$ be another random vector. Then, 
		\begin{equation}
			Cov(\textbf{X, Y}) = E({(\textbf{X} - \mu_\textbf{X})}^T(\textbf{Y} - \mu_\textbf{Y})) = [\Sigma]_{n \times m}
		\end{equation}
	
		\subsection{Covariance}
			Covariance means how two things(random variables $X_1$ and $X_2$) vary with respect to each other. If on increase of $X_1$, $X_2$ increases and on decrease of $X_1$, $X_2$ decreases, we say covariance is positive. If on increase of $X_1$, $X_2$ decreases and vice versa, we say that covariance is negative. Covariance is 0 if no such relation exists. Understanding the magnitude of covariance is difficult. 
			\\
			Mathematically, let $X_1$ and $X_2$ are two real valued random variables. Then Cov($X_1$, $X_2$) = $E[X_1 - \mu_1)\times(X_2-\mu_2)]$
		
		
		\subsection{Properties of Covariance Matrices}
			\begin{itemize}
				\item Positive semidefinite
				\item Symmetric
				\item If ($m = n$)
			\end{itemize}


	\section{Random Variable}
		Random variable is a function that maps outcomes of an event to mathematically convenient form (real numbers). For example: Let our event be "\textit{A coin is tossed 30 times}" and let random variable be \textbf{Number of heads occured while tossing 30 times}
		
		\subsection{Random Vector}
			Random vector are used when you want to view multiple events simultaneously. For example: You want to observe tossing of coin ($X_1$) and rolling of dice ($X_2$) simultaneously. Then, you define random vector $\textbf{X} = \{X_1, X_2\}$. Here $X_1: {-1,1}$ and $X_2: {1,2,3,4,5,6}$. In more machine learning pov, you have random vector as your feature set where each feature is a random variable.

	\section{PCA}
	
\end{document}